\documentclass{article} % This command is used to set the type of document you are working on such as an article, book, or presenation

\usepackage{geometry} % This package allows the editing of the page layout
\usepackage{amsmath}  % This package allows the use of a large range of mathematical formula, commands, and symbols
\usepackage{graphicx}  % This package allows the importing of images

\newcommand{\question}[2][]{\begin{flushleft}\textbf{Question #1}: \textit{#2}\end{flushleft}}
\newcommand{\sol}{\textbf{Solution}:} %Use if you want a boldface solution line
\newcommand{\maketitletwo}[2][]{\begin{center}
        \Large{\textbf{Lab 2 Report}
        
            Deep Learning} % Name of course here
        \vspace{5pt}
        
        \normalsize{
            Name: Kai-Jie Lin 
            
            Student ID: 110652019
            
            \today}
        \vspace{15pt}
        \end{center}}
\begin{document}
    \maketitletwo[5]  % Optional argument is assignment number
    %Keep a blank space between maketitletwo and \question[1]
    
    \section{Introduction}

    In this lab, I implemented VGG19 and ResNet50 architecture with Pytorch.
    I use the models to classify the butterflies and moths dataset with 100 classes.
    Futhermore, I designed my own dataloader and data preprocessing technique to train the model.
    Finally, I evaluated the model with the test dataset and calcßlated the accuracy.

    \section{Implementation Details}
    \subsection{Details of models}
    \textbf{VGG19} \\
    Optimizer: SGD \\
    Criterion: CrossEntropyLoss \\
    Model architecture: \\
    \includegraphics[width=7cm]{img/vgg1.png}
    \includegraphics[width=7cm]{img/vgg2.png}
    \\ \textbf{ResNet50} \\
    Optimizer: SGD \\
    Criterion: CrossEntropyLoss \\
    BottleNeckBlock: \\
    \includegraphics[width=12cm]{img/Bottleneck.png} \\
    Model architecture: \\
    \includegraphics[width=9cm]{img/resnet.png} \\

    \subsection{Details of dataloader}
    \textbf{Dataloaders} \\
    I designed my own dataloader to load the dataset. Here batch size is 32. \\
    \includegraphics[width=12cm]{img/loaders.png} \\
    get\_item function: \\
    \includegraphics[width=9cm]{img/dataloader.png} \\

    \section{Data Preprocessing}
    \subsection{How to preprocess the data?}
    I use some data augmentation techniques to preprocess the data. 
    For example, random flip the image and randomly change the brightness, contrast, saturation and hue of an image.\\
    \includegraphics[width=14cm]{img/augmentation.png} \\
    Original image: \\
    \includegraphics[width=5cm]{img/orig.png} \\
    Augmented image: \\
    \includegraphics[width=5cm]{img/aug.png} \\
    \subsection{What makes my method special?}
    I use the data augmentation technique to increase the diversity of the dataset. 
    This can help the model to learn more features and improve the accuracy.

    \section{Experimental Results}
    \subsection{The highest accuracy}
    \textbf{VGG19} \\
    The highest accuracy on test dataset is 88.8\%. \\
    \includegraphics[width=14cm]{img/vgg_score.png} \\
    \textbf{ResNet50} \\
    The highest accuracy on test dataset is 88.4\%. \\
    \includegraphics[width=14cm]{img/resnet_score.png} \\
    \subsection{Comparison}
    \textbf{Training Loss}: \\
    \includegraphics[width=8cm]{img/loss.png} \\
    \textbf{Training and Validation Accuracy per Epoch}: \\
    \includegraphics[width=8cm]{img/acc.png} \\
    \textbf{Test on each datasets}: \\
    \includegraphics[width=14cm]{img/test_vgg.png} \\
    \includegraphics[width=14cm]{img/test_resnet.png} \\
    \section{Discussion}
    \subsection{Different Optimizer}
    I compare the performance of the model with SGD optimizer and Adam optimizer.
    The result shows that the model with Adam optimizer has a higher accuracy under test dataset than the model with SGD optimizer.
    SGD can reach higher accuracy on training dataset but Adam can reach higher accuracy on test dataset.
    SGD is more likely to overfit the training dataset. \\
    \textbf{Training Loss}: \\
    \includegraphics[width=8cm]{img/loss_opt.png} \\
    \textbf{Training and Test Accuracy per Epoch}: \\
    \includegraphics[width=8cm]{img/acc_opt.png} \\
    

\end{document}